%%%%%%%%%%%%%%%%%%%%%%%%%%%%%%%%%%%%%%%%%%%%%%%%%%%%%%%%%%%%%%%%%%%%%%
% Experimental code for doing a sanity check - to see what default values are still in the document
% these should have been replaced by the author.
%


\makeatletter
% --- Expl3 Definitions for SDG Range Check ---
\ExplSyntaxOn

% Declare expl3 variables for the check process
\int_new:N \l_kththesis_check_min_int      % Stores the minimum allowed value
\int_new:N \l_kththesis_check_max_int      % Stores the maximum allowed value
\int_new:N \l_kththesis_current_item_int   % Stores the current item from the list as an integer
\seq_new:N \g_kththesis_invalid_sdgs_seq    % Stores items from the list that are out of range
\bool_new:N \g_kththesis_all_sdgs_valid_bool % Flag to indicate if all items were within range


% --- Main function to check SDG range ---
% #1 = comma-separated list of SDGs (e.g., "1,9")
% #2 = minimum allowed value (e.g., 1)
% #3 = maximum allowed value (e.g., 17)
\cs_new_protected:Npn \kththesis_check_sdg_range:nnn #1#2#3 {
    %checking~"#1"~"#2"~"#3"\par
    \typeout{checking~#1:~Actual="#1",~Min=#2,~Max=#3} % Debugging: shows in .log file
    % Set effective min/max values for comparison (min-1 and max+1)
    % Note that we have to decrement the first and increment the last
    % as the  \int_compare_p:nNn comparison relations are < and > -- rather than <= and >=
    \int_set:Nn \l_kththesis_check_min_int {#2}
    \int_decr:N \l_kththesis_check_min_int              % Becomes min-1
    \int_set:Nn \l_kththesis_check_max_int {#3}
    \int_incr:N \l_kththesis_check_max_int              % Becomes max+1
    
    % Clear results from previous runs of this function
    \seq_gclear:N \g_kththesis_invalid_sdgs_seq
    \bool_gset_true:N \g_kththesis_all_sdgs_valid_bool % Assume all items are valid until proven otherwise
    \seq_set_split:Nne \l_tmpa_seq { , }  { #1 } % 'e' expansion to expand #1 before splitting
    
    %~value~is:~\seq_use:Nnnn \l_tmpa_seq { ~and~ } { ,~ } { ,~and~ } \par
    % Iterate through each item in the sequence
    \seq_map_variable:NNn \l_tmpa_seq \l_tmpa_tl {
        % For robust error handling check if the input is not a number
        % --- CRITICAL FIX: Check if the item is numeric first ---
        \IfInteger{\l_tmpa_tl}
        { % True: Item is numeric, proceed with comparison
            %l_tmpa_tl~is:~\tl_use:N {\l_tmpa_tl}\par
            \int_set:Nn \l_kththesis_current_item_int {\l_tmpa_tl}
            % Debugging output for each item
            \typeout{ -- Checking~item: \tl_use:N \l_tmpa_tl\ (int: \int_use:N \l_kththesis_current_item_int)}
            % Compare the current item with the min/max range
            % \bool_lazy_and:nTF checks if (item > min-1) AND (item < max+1)
            \bool_lazy_and:nnTF
                { \int_compare_p:nNn { \l_kththesis_current_item_int } > { \l_kththesis_check_min_int } }
                { \int_compare_p:nNn { \l_kththesis_current_item_int } < { \l_kththesis_check_max_int } }
            { % True branch: Item is within range (do nothing, it's valid)
                \typeout{~~~~--~Item~is~within~range.}
            }
            { % False branch: Item is OUT of range
                \typeout{~~~~--~Item~is~OUT~of~range!}
                \bool_gset_false:N \g_kththesis_all_sdgs_valid_bool % Mark the overall check as having invalid items
                \seq_gput_right:Ne \g_kththesis_invalid_sdgs_seq {\l_tmpa_tl} % Add the _value_ of the invalid item to the sequence
            }
        }{
        % False: Item is NOT numeric
            \typeout{ ~-~Checking~item:~\tl_use:N \l_tmpa_tl\ (non-numeric)}
            \bool_gset_false:N \g_kththesis_all_sdgs_valid_bool % Mark overall check as invalid
            \seq_gput_right:Ne \g_kththesis_invalid_sdgs_seq { \l_tmpa_tl } % Add non-numeric item to invalid list
        }
    }
    
}

% --- Wrapper macro for easy calling from standard LaTeX mode ---
% This macro is what you will use in your document.
% #1 = The comma-separated list of SDGs (e.g., from \@SDGs)
% #2 = Minimum allowed value
% #3 = Maximum allowed value
\newcommand{\checkSDGRange}[3]{
    \ExplSyntaxOn % Activate expl3 syntax for the call to the expl3 function
    \tl_if_empty:eTF {#1} {
    Empty~list
    }
    {
        % Call the expl3 core function to perform the check
        \kththesis_check_sdg_range:nnn {#1}{#2}{#3}
    
        % Report results based on the flags set by the expl3 function
        \bool_if:NTF \g_kththesis_all_sdgs_valid_bool
        { % All items were valid
            % The following informational message (if uncommented) will appear at the end of the PDF file
            %All~SDGs~are~within~the~range~[#2-#3].
        }
        { % Some items were invalid
            % The following informational message (if uncommented) will appear at the end of the PDF file
            %The~following~SDGs~are~out~of~range~[#2-#3]:~\seq_use:Nn \g_kththesis_invalid_sdgs_seq {,~} .
            \PackageWarning{kththesis}{The~following~SDGs~are~out~of~range~[#2-#3]:~\seq_use:Nn \g_kththesis_invalid_sdgs_seq {,~}.~Detected~at~end~of~document~}%
        }
    }
    \ExplSyntaxOff % Deactivate expl3 syntax
}

\newcommand{\checkSDGs}[0]{
\checkSDGRange{\@SDGs}{1}{17}
}
\ExplSyntaxOff % End of all expl3 definitions

% --- Expl3 Definitions for National Subject Categories Check ---
\ExplSyntaxOn

% Declare expl3 variables for the check process
\seq_new:N \g_kththesis_invalid_NSCs_seq    % Stores items from the list that are out of range
\bool_new:N \g_kththesis_all_NSCs_valid_bool % Flag to indicate if all items were within range

\seq_const_from_clist:Nn \c_kththesis_all_NSCs_seq { 1,101,10101,10102,10103,10104,10105,10106,10199,102,10201,10202,10203,10204,10205,10206,10207,10208,10210,10211,10212,10213,10214,10299,103,10301,10302,10303,10304,10305,10307,10308,10399,104,10401,10402,10403,10404,10405,10406,10407,10408,10499,105,10501,10502,10503,10504,10505,10506,10507,10508,10509,10510,10599,106,10601,10604,10605,10606,10607,10608,10609,10610,10611,10612,10613,10614,10615,10616,10699,107,10799,2,201,20101,20102,20103,20104,20105,20106,20107,20109,20110,20199,202,20201,20202,20203,20204,20205,20206,20207,20208,20209,20299,203,20301,20302,20304,20305,20306,20307,20309,20310,20399,204,20402,20403,20405,20406,20407,20499,205,20501,20502,20503,20504,20505,20506,20599,206,20601,20602,20603,20604,20605,20606,20699,207,20702,20703,20704,20705,20707,20799,208,20801,20802,20803,20899,209,20901,20902,20903,20904,20905,20906,20909,20999,210,21002,21003,21004,21005,21099,211,21199,3,301,30101,30102,30103,30104,30105,30106,30107,30108,30109,30110,30111,30112,30113,30114,30115,30116,30117,30118,30199,302,30201,30202,30203,30204,30205,30206,30207,30208,30209,30211,30212,30213,30215,30216,30217,30218,30219,30220,30221,30222,30223,30224,30225,30226,30227,30228,30229,30230,30299,303,30301,30303,30304,30305,30306,30307,30308,30309,30310,30311,30312,30313,30314,30399,304,30401,30402,30403,30499,305,30501,30502,30599,4,401,40101,40102,40103,40104,40105,40106,40107,402,40201,403,40301,40302,40303,40399,404,40401,40402,405,40502,40504,40505,40506,40507,40599,5,501,50101,50102,502,50201,50202,50203,503,50301,50302,50304,50399,504,50401,50402,50404,50405,50406,505,50501,50503,506,50601,50604,507,50701,50702,50703,508,50801,50804,50805,509,50902,50903,50904,50905,50906,50907,50908,50909,50910,50911,50912,50999,6,601,60101,60102,60103,60104,60105,602,60201,60202,60203,60204,60205,60206,60207,603,60301,60302,60303,60304,60306,604,60407,60408,60409,60410,60411,60412,60413,60414,60415,60416,60417,60418,60419,605,60502,60503,60504,60599 }


% --- Main function to check SDG range ---
% #1 = comma-separated list of SDGs (e.g., "1,9")
\cs_new_protected:Npn \kththesis_check_NSCs:n #1 {
    %checking~"#1"\par
    \typeout{checking~#1:~Actual="#1"} % Debugging: shows in .log file

    % Clear results from previous runs of this function
    \seq_gclear:N \g_kththesis_invalid_NSCs_seq
    \bool_gset_true:N \g_kththesis_all_NSCs_valid_bool % Assume all items are valid until proven otherwise
    \seq_set_split:Nne \l_tmpa_seq { , }  { #1 } % 'e' expansion to expand #1 before splitting
    
    %~value~is:~\seq_use:Nnnn \l_tmpa_seq { ~and~ } { ,~ } { ,~and~ } \par
    % Iterate through each item in the sequence
    \seq_map_variable:NNn \l_tmpa_seq \l_tmpa_tl {
        % For robust error handling check if the input is not a number
        % --- CRITICAL FIX: Check if the item is numeric first ---
        \IfInteger{\l_tmpa_tl}
        { % True: Item is numeric, proceed with comparison
            % Compare the current item with the valid values
            \seq_if_in:NVTF \c_kththesis_all_NSCs_seq {\l_tmpa_tl}
            { % True branch: Item is within range (do nothing, it's valid)
                \typeout{~~~~--~Item~is~valid.}
            }
            { % False branch: Item is invalid
                \typeout{~~~~--~Item~is~invalid!}
                \bool_gset_false:N \g_kththesis_all_NSCs_valid_bool % Mark the overall check as having invalid items
                \seq_gput_right:Ne \g_kththesis_invalid_NSCs_seq {\l_tmpa_tl} % Add the _value_ of the invalid item to the sequence
            }
        }{
        % False: Item is NOT numeric
            \typeout{ ~-~Checking~item:~\tl_use:N \l_tmpa_tl\ (non-numeric)}
            \bool_gset_false:N \g_kththesis_all_NSCs_valid_bool % Mark overall check as invalid
            \seq_gput_right:Ne \g_kththesis_invalid_NSCs_seq { \l_tmpa_tl } % Add non-numeric item to invalid list
        }
    }
    
}

% --- Wrapper macro for easy calling from standard LaTeX mode ---
% This macro is what you will use in your document.
% #1 = The comma-separated list of SDGs (e.g., from \@SDGs)
\newcommand{\checkNSCs}[1]{
    \ExplSyntaxOn % Activate expl3 syntax for the call to the expl3 function
    \tl_if_empty:eTF {#1} {
    Empty~list
    }
    {
        % Call the expl3 core function to perform the check
        \kththesis_check_NSCs:n {#1}
    
        % Report results based on the flags set by the expl3 function
        \bool_if:NTF \g_kththesis_all_NSCs_valid_bool
        { % All items were valid
            %All~national~subject~categories~are~valid.
        }
        { % Some items were invalid
            %The~following~national~subject~categories~are~invalid:~\seq_use:Nn \g_kththesis_invalid_NSCs_seq {,~} .
            \PackageWarning{kththesis}{The~following~national~subject~categories~are~invalid:~\seq_use:Nn \g_kththesis_invalid_NSCs_seq {,~}.~Detected~at~end~of~document~}%
        }
    }
    \ExplSyntaxOff % Deactivate expl3 syntax
}

\newcommand{\checknationalsubjectcategories}[0]{
\checkNSCs{\@nationalsubjectcategories}
}
\ExplSyntaxOff % End of all expl3 definitions

% --- Expl3 Definitions for Sanity Check ---
% All expl3 commands and variable declarations must be within \ExplSyntaxOn / \ExplSyntaxOff.
\ExplSyntaxOn

% --- Global Counters and Sequence for Sanity Check Results ---
\int_new:N \g_kth@defaultCount_int       % Counts values that are still at default
\int_new:N \g_kth@totalCheckedCount_int   % Counts total values checked
\seq_new:N \g_kth@unchangedValues_seq     % Stores names of values that are still at default

% --- Helper Macro to Compare a Value with its Default (expl3 syntax) ---
% #1 = A descriptive name for the value (e.g., "Author Name")
% #2 = The actual stored value (will be expanded via 'e' type)
% #3 = The default value (will be expanded via 'e' type)
\cs_new_protected:Npn \kth@check@value:nnn #1#2#3 {
    \typeout{Checking #1: Actual="#2", Default="#3"} % Debugging: shows in .log file
    \int_gincr:N \g_kth@totalCheckedCount_int % Increment global total count
    \str_if_eq:eeTF {#2}{#3} % Robust string comparison (expands both arguments 'e')
    { % True branch: Value is still at default
        \typeout{ -- Still at default value} % Debugging
        \int_gincr:N \g_kth@defaultCount_int % Increment global default count
        \seq_gput_right:Nn \g_kth@unchangedValues_seq {#1} % Add descriptive name to global sequence
    }
    { % False branch: Value has been changed
        \typeout{ -- Value changed (OK)} % Debugging
        % Do nothing, as per requirement "warning message per unchanged value"
    }
}
 
% --- The Main Sanity Check Macro (expl3 version) ---
% This macro runs the checks and issues warnings.
\cs_new_protected:Npn \kth@perform@sanity@check_expl3: {
    % Reset counters and sequence for each run
    \int_gzero:N \g_kth@defaultCount_int
    \int_gzero:N \g_kth@totalCheckedCount_int
    \seq_gclear:N \g_kth@unchangedValues_seq

    % --- Call expl3 helper for each item to check ---
    % Pass the content of the standard LaTeX macros directly.

    % Added checks
    \kth@check@value:nnn {courseCycle}{\@courseCycle}{XXX}
    \kth@check@value:nnn {courseCode}{\@courseCode}{XX100X}
    \kth@check@value:nnn {courseCredits}{\@courseCredits}{XXX}
    \kth@check@value:nnn {programcode}{\@programcode}{TXXXX}
    \kth@check@value:nnn {degreeName}{\@degreeName}{*****Unknown~degree*****}
    \kth@check@value:nnn {subjectArea}{\@subjectArea}{*****Unknown~subject~area*****}
    
    % title related
    \kth@check@value:nnn {title}{\@title}{This~is~the~title~in~the~language~of~the~thesis}
    \kth@check@value:nnn {subtitle}{\@subtitle}{A~subtitle~in~the~language~of~the~thesis}
    \ifinswedish
    \kth@check@value:nnn {alttitle}{\@alttitle}{This is the English translation of the title}
    \kth@check@value:nnn {altsubtitle}{\@altsubtitle}{This is the English translation of the subtitle}
    \else
    \kth@check@value:nnn {alttitle}{\@alttitle}{Detta~är~den~svenska~översättningen~av~titeln}
    \kth@check@value:nnn {altsubtitle}{\@altsubtitle}{Detta~är~den~svenska~översättningen~av~undertiteln}
    \fi

    % --- Author Information ---
    \kth@check@value:nnn {authorsLastname}{\@authorsLastname}{Student}
    \kth@check@value:nnn {authorsFirstname}{\@authorsFirstname}{Fake~A.}
    \kth@check@value:nnn {email}{\@email}{XXXXXXXXXXXX@kth.se}
    \kth@check@value:nnn {kthid}{\@kthid}{u1XXXXXX}
    \kth@check@value:nnn {authorsSchool}{\@authorsSchool}{\schoolAcronym{XXX}}

    \tl_if_empty:NTF \@hostcompany
    { % nothing to do - there is no host company
    }
    { % otherwise, check details of host company
    \kth@check@value:nnn {hostcompany}{\@hostcompany}{Företaget~AB}
    }

    \tl_if_empty:NTF \@hostorganization
    { % nothing to do - there is no host organization
    }
    { % otherwise, check details of host organization
    \kth@check@value:nnn {hostorganization}{\@hostorganization}{XXXNationalLab}
    }
    
    % If there is a second author - add them here:
    \tl_if_empty:NTF \@secondAuthorsLastname
    { % nothing to do - there is no second author
    }
    { % otherwise, check details of second author
    \kth@check@value:nnn {secondAuthorsLastname}{\@secondAuthorsLastname}{Student}
    \kth@check@value:nnn {secondAuthorsFirstname}{\@secondAuthorsFirstname}{Fake~B.}
    \kth@check@value:nnn {secondemail}{\@secondemail}{XXXXXXXXXXXX@kth.se}
    \kth@check@value:nnn {secondkthid}{\@secondkthid}{u1XXXXXX}
    \kth@check@value:nnn {secondAuthorsSchool}{\@secondAuthorsSchool}{\schoolAcronym{XXX}}
    }

    % --- Supervisor Information ---
    \tl_if_empty:NTF \@supervisorAsLastname
    { % nothing to do - there is no second supervisor
    }
    { % otherwise, check details of first supervisor
    \kth@check@value:nnn {supervisorAsLastname}{\@supervisorAsLastname}{Supervisor}
    \kth@check@value:nnn {supervisorAsFirstname}{\@supervisorAsFirstname}{A.~Busy}
    \kth@check@value:nnn {supervisorAsEmail}{\@supervisorAsEmail}{XXXXXXXXXXXX@kth.se}
    \kth@check@value:nnn {supervisorAsKTHID}{\@supervisorAsKTHID}{u1XXXXXX}
    \kth@check@value:nnn {supervisorAsSchool}{\@supervisorAsSchool}{\schoolAcronym{XXX}}
    \kth@check@value:nnn {supervisorAsDepartment}{\@supervisorAsDepartment}{XXX}
    }
    
    % If there is a second supervisor add them here:
    \tl_if_empty:NTF \@supervisorBsLastname
    { % nothing to do - there is no second supervisor
    }
    { % otherwise, check details of second supervisor
    \kth@check@value:nnn {supervisorBsLastname}{\@supervisorBsLastname}{Supervisor}
    \kth@check@value:nnn {supervisorBsFirstname}{\@supervisorBsFirstname}{Another~Busy}
    \kth@check@value:nnn {supervisorBsEmail}{\@supervisorBsEmail}{XXXXXXXXXXXX@kth.se}
    \kth@check@value:nnn {supervisorBsKTHID}{\@supervisorBsKTHID}{u1XXXXXX}
    \kth@check@value:nnn {supervisorBsSchool}{\@supervisorBsSchool}{\schoolAcronym{XXX}}
    \kth@check@value:nnn {supervisorBsDepartment}{\@supervisorBsDepartment}{XXX}
    }

    % If there is a third supervisor add them here:
    \tl_if_empty:NTF \@supervisorCsLastname
    { % nothing to do - there is no third supervisor
    }
    { % otherwise, check details of third supervisor
    \kth@check@value:nnn {supervisorCsLastname}{\@supervisorCsLastname}{Supervisor}
    \kth@check@value:nnn {supervisorCsFirstname}{\@supervisorCsFirstname}{Third~Busy}
    \kth@check@value:nnn {supervisorCsEmail}{\@supervisorCsEmail}{XXXXXXXXXXXX@tu.va}
    \kth@check@value:nnn {supervisorCsOrganization}{\@supervisorCsOrganization}{Timbuktu~University,~Department~of~Pseudoscience}
    }

    % --- Examiner Information ---
    \kth@check@value:nnn {examinersLastname}{\@examinersLastname}{ExaminersLastname}
    \kth@check@value:nnn {examinersFirstname}{\@examinersFirstname}{ExaminersFirstname}
    \kth@check@value:nnn {examinersEmail}{\@examinersEmail}{XXXXXXXXXXXX@kth.se}
    \kth@check@value:nnn {examinersKTHID}{\@examinersKTHID}{u1XXXXXX}
    \kth@check@value:nnn {examinersSchool}{\@examinersSchool}{\schoolAcronym{XXX}}
    \kth@check@value:nnn {examinersDepartment}{\@examinersDepartment}{XXX}

    %  --- SDG and National Subject Categories Information ---
    \kth@check@value:nnn {nationalsubjectcategories}{\@nationalsubjectcategories}{ddddd,~ddddd}
    \kth@check@value:nnn {SDGs}{\@SDGs}{XXX,~XXX}

    % --- Keywords ---
    \kth@check@value:nnn {EnglishKeywords}{\@EnglishKeywords}{KeywordA,~KeywordB,~KeywordC}
    \kth@check@value:nnn {SwedishKeywords}{\@SwedishKeywords}{NyckelordA,~NyckelordB,~NyckelordC}

    %  --- Name of opponent or opponents ---
    % If it not empty and not the default, then we check the other presentation
    % related parameters to see if they have their default values
    \tl_if_empty:NTF \@opponentsNames
    { % nothing to do - there is no opponent named
    }
    {
        \kth@check@value:nnn {opponentsNames}{\@opponentsNames}{XXX}
        \str_compare:eNeTF {\@opponentsNames} = {XXX}
        { % Nothing to do - this is the set default
        }
        { % if False - check the other presentation values for defaults
            \kth@check@value:nnn {presentationDateAndTimeISO}{\@presentationDateAndTimeISO}{2022-03-15~13:00}
            \kth@check@value:nnn {presentationLanguage}{\@presentationLanguage}{XXX}
            % \{eng}
            \kth@check@value:nnn {presentationDateAndTimeISO}{\@presentationDateAndTimeISO}{2022-03-15~13:00}
            \kth@check@value:nnn {presentationRoom}{\@presentationRoom}{via~Zoom~https://kth-se.zoom.us/j/ddddddddddd}
            \kth@check@value:nnn {presentationAddress}{\@presentationAddress}{Isafjordsgatan~22~(Kistagången~16)}
            % \presentationCity{Stockholm}
        }
    }


    % --- Issue Warnings based on results (using expl3 variables and sequence) ---
    \int_compare:nNnTF { \g_kth@totalCheckedCount_int } = { 0 }
    { % No values were checked (e.g., if list of checks is empty)
        % No warning needed
    }
    { % Values were checked
        \int_compare:nNnTF { \g_kth@defaultCount_int } = { \g_kth@totalCheckedCount_int } % All values are at default
        { % All values are at default
            \PackageWarning{kththesis}{The~state~of~the~project~is~the~default~configuration,~please~customize~it~for~yourself.~Detected~at~end~of~document~}
        }
        { % Some values are default, some are changed
            \int_compare:nNnTF { \g_kth@defaultCount_int } > { 0 } % Only if there are any unchanged values
            { % There are some unchanged values. Use \seq_use:Nnnn for formatted list.
              % Nnnn means it uses the sequence, then a separator for the last two items,
              % a separator for all items, and a separator for the last two items in a list of 3+.
              % Here: {item1, item2 and item3}
                \PackageWarning{kththesis}{Some~values~are~still~set~at~their~default~value.~Please~review~the~following:~\seq_use:Nnnn \g_kth@unchangedValues_seq {~and~}{,~}{,~and~}.~Detected~at~end~of~document~}%
            }
            { % No unchanged values left (all changed)
                % No action
            }
        }
    }
    \@ifpackagelater{expl3}{2024-03-14}{%
        \checkSDGs{}
    }{%
        % need the expl3 package as of TeX Live 2024 for
        % \tl_if_empty:eTF and \seq_set_split:Nne
        % USed when checking SDGs
    }
    \checknationalsubjectcategories{}
}
\ExplSyntaxOff % End of expl3 definitions

% --- Standard LaTeX wrapper macro for calling the expl3 helper ---
% This makes the expl3-based sanity check callable from standard LaTeX mode.
\ExplSyntaxOn
\NewDocumentCommand{\kthPerformSanityCheck}{}
{%
    \kth@perform@sanity@check_expl3: % Call the expl3 helper (note the colon)
}
\ExplSyntaxOff % Deactivate expl3 syntax
\makeatother

\AtEndDocument{\kthPerformSanityCheck}
