%\NeedsTeXFormat{LaTeX2e}[1999/12/01]
%\ProvidesPackage{kth-commands}
%    [2025/10/09 v0.1 Provides template-specific commands needed for kththesis.cls]

\makeatletter
%% Macros from the package phfnote
% \phfnoteSaveDefs{〈identifier〉}{〈list of macro names〉} saves the current definitions of the given list of macros under the identifier. The list of macros is specified as a comma-separated list of macro names. Note that there cannot be spaces after the commas in the list.

\def\phfnoteSaveDefs#1#2{%
  \csgdef{phfnote@restoredefs@#1}{}%
  \def\@tmpa{#2}%
  \@for\next:=\@tmpa\do{%
    \global\csletcs{phfnote@restoredefs@#1@\next}{\next}%
    \expandafter\xappto\csname phfnote@restoredefs@#1\endcsname{%
      \noexpand\csletcs{\next}{phfnote@restoredefs@#1@\next}%
    }%
  }%
}

% \phfnoteRestoreDef{〈identifier〉} - restores the saved macros
\def\phfnoteRestoreDefs#1{%
  \ifcsname phfnote@restoredefs@#1\endcsname%
    \csname phfnote@restoredefs@#1\endcsname%
  \else%
    \PackageError{phfnote}{\string\phfnoteRestoreDefs: no such
      definitions stored (#1)}{}
  \fi%
}

%% To give warnings about the use of citations, we set a hook before the abstract and one after it
%% The first saves the definition of \cite and sets up a warning while the second hook restores it
\BeforeBeginEnvironment{abstract}{\phfnoteSaveDefs{origcmds}{cite,ref,Cref,cref}%
\renewcommand{\cite}[1]{{\NotoSansFont \textcolor{red}{⁉\uuline{cite}⁉}}\PackageWarning{kththesis}{attempt to use cite command in an abstract}}%
\renewcommand{\ref}[1]{{\NotoSansFont \textcolor{red}{⁉\uuline{ref}⁉}}\PackageWarning{kththesis}{attempt to use ref command in an abstract}}%
\renewcommand{\Cref}[1]{{\NotoSansFont \textcolor{red}{⁉\uuline{Cref}⁉}}\PackageWarning{kththesis}{attempt to use Cref command in an abstract}}%
\renewcommand{\cref}[1]{{\NotoSansFont \textcolor{red}{⁉\uuline{cref}⁉}}\PackageWarning{kththesis}{attempt to use cref command in an abstract}}%
}

\AfterEndEnvironment{abstract}{\phfnoteRestoreDefs{origcmds}}

\makeatletter
% Define commands for setting the user definable parts of the title page

\let\@subtitle\@empty
\newcommand{\subtitle}[1]{\def\@subtitle{#1}}
\let\@alttitle\@empty
\newcommand{\alttitle}[1]{\def\@alttitle{#1}}
\let\@altsubtitle\@empty
\newcommand{\altsubtitle}[1]{\def\@altsubtitle{#1}}

% enable the author to specify plain text titles and subtitles to handle the situation when the titles and subtitles cannot be translated to pure Unicode

\let\@titleInPlainText\@empty
\newcommand{\titleInPlainText}[1]{\def\@titleInPlainText{#1}}
\let\@subtitleInPlainText\@empty
\newcommand{\subtitleInPlainText}[1]{\def\@subtitleInPlainText{#1}}
\let\@alttitleInPlainText\@empty
\newcommand{\alttitleInPlainText}[1]{\def\@alttitleInPlainText{#1}}
\let\@altsubtitleInPlainText\@empty
\newcommand{\altsubtitleInPlainText}[1]{\def\@altsubtitleInPlainText{#1}}


\let\@hostcompany\@empty
\newcommand{\hostcompany}[1]{\def\@hostcompany{#1}}
\let\@hostorganization\@empty
\newcommand{\hostorganization}[1]{\def\@hostorganization{#1}}

\let\@school\@empty
\newcommand{\school}[1]{\def\@school{#1}}

% First author's information
\let\@authorsLastname\@empty
\newcommand{\authorsLastname}[1]{\def\@authorsLastname{#1}}
\let\@authorsFirstname\@empty
\newcommand{\authorsFirstname}[1]{\def\@authorsFirstname{#1}}
\let\@email\@empty
\newcommand{\email}[1]{\def\@email{#1}}
\let\@kthid\@empty
\newcommand{\kthid}[1]{\def\@kthid{#1}}
%\let\@orcid\@empty
%\newcommand{\orcid}[1]{\def\@orcid{#1}}
\let\@authorsSchool\@empty
\newcommand{\authorsSchool}[1]{\def\@authorsSchool{#1}}
\let\@authorsDepartment\@empty
\newcommand{\authorsDepartment}[1]{\def\@authorsDepartment{#1}}
% Information about the city and country for the acknowledgement
\let\@authorCity\@empty
\newcommand{\authorCity}[1]{\def\@authorCity{#1}}
\let\@authorCountry\@empty
\newcommand{\authorCountry}[1]{\def\@authorCountry{#1}}
\let\@authorCityCountryDate\@empty
% Define the command so that if a date is not specified it will not output the comma; it only outputs the comma if there is a date.
\ExplSyntaxOn
\cs_set_eq:NN  \IfEmptyTF  \tl_if_blank:nTF
\ExplSyntaxOff
\NewDocumentCommand{\authorCityCountryDate}{m}{
\def\@authorCityCountryDate{\ifx\@authorCity\@empty%
    Stockholm%
    \else%
    \ifx\@authorCountry\@empty%
    \@authorCity%
    \else%
    \@authorCity, \@authorCountry%
    \fi%
    \fi%
    \IfEmptyTF{#1}{}{, #1}%
    }%
}
\authorCityCountryDate{\MONTH\enspace\the\year} % default month and year for the acknowledgement signature

% Second author's information
\let\@secondAuthorsLastname\@empty
\newcommand{\secondAuthorsLastname}[1]{\def\@secondAuthorsLastname{#1}}
\let\@secondAuthorsFirstname\@empty
\newcommand{\secondAuthorsFirstname}[1]{\def\@secondAuthorsFirstname{#1}}
\let\@secondemail\@empty
\newcommand{\secondemail}[1]{\def\@secondemail{#1}}
\let\@secondkthid\@empty
\newcommand{\secondkthid}[1]{\def\@secondkthid{#1}}
%\let\@secondorcid\@empty
%\newcommand{\secondorcid}[1]{\def\@secondorcid{#1}}
\let\@secondAuthorsSchool\@empty
\newcommand{\secondAuthorsSchool}[1]{\def\@secondAuthorsSchool{#1}}
\let\@secondAuthorsDepartment\@empty
\newcommand{\secondAuthorsDepartment}[1]{\def\@secondAuthorsDepartment{#1}}
% Information about the city and country for the acknowledgement
\let\@secondAuthorCity\@empty
\newcommand{\secondAuthorCity}[1]{\def\@secondAuthorCity{#1}}
\let\@secondAuthorCountry\@empty
\newcommand{\secondAuthorCountry}[1]{\def\@secondAuthorCountry{#1}}
\let\@secondAuthorCityCountryDate\@empty
% Define the command so if a date is not specified it will not output the comma; it only outputs the comma if there is a date.
\NewDocumentCommand{\secondAuthorCityCountryDate}{m}{
\def\@secondAuthorCityCountryDate{\ifx\@secondAuthorCity\@empty%
    Stockholm%
    \else%
    \ifx\@secondAuthorCountry\@empty%
    \@secondAuthorCity%
    \else%
    \@secondAuthorCity, \@secondAuthorCountry%
    \fi%
    \fi%
    \IfEmptyTF{#1}{}{, #1}%%
    }%
}

\let\@supervisorAsLastname\@empty
\newcommand{\supervisorAsLastname}[1]{\def\@supervisorAsLastname{#1}}
\let\@supervisorAsFirstname\@empty
\newcommand{\supervisorAsFirstname}[1]{\def\@supervisorAsFirstname{#1}}
\let\@supervisorAsEmail\@empty
\newcommand{\supervisorAsEmail}[1]{\def\@supervisorAsEmail{#1}}
\let\@supervisorAsKTHID\@empty
\newcommand{\supervisorAsKTHID}[1]{\def\@supervisorAsKTHID{#1}}
\let\@supervisorAsOrganization\@empty
\newcommand{\supervisorAsOrganization}[1]{\def\@supervisorAsOrganization{#1}}
\let\@supervisorAsSchool\@empty
\newcommand{\supervisorAsSchool}[1]{\def\@supervisorAsSchool{#1}}
\let\@supervisorAsDepartment\@empty
\newcommand{\supervisorAsDepartment}[1]{\def\@supervisorAsDepartment{#1}}

\let\@supervisorBsLastname\@empty
\newcommand{\supervisorBsLastname}[1]{\def\@supervisorBsLastname{#1}}
\let\@supervisorBsFirstname\@empty
\newcommand{\supervisorBsFirstname}[1]{\def\@supervisorBsFirstname{#1}}
\let\@supervisorBsEmail\@empty
\newcommand{\supervisorBsEmail}[1]{\def\@supervisorBsEmail{#1}}
\let\@supervisorBsKTHID\@empty
\newcommand{\supervisorBsKTHID}[1]{\def\@supervisorBsKTHID{#1}}
\let\@supervisorBsOrganization\@empty
\newcommand{\supervisorBsOrganization}[1]{\def\@supervisorBsOrganization{#1}}
\let\@supervisorBsSchool\@empty
\newcommand{\supervisorBsSchool}[1]{\def\@supervisorBsSchool{#1}}
\let\@supervisorBsDepartment\@empty
\newcommand{\supervisorBsDepartment}[1]{\def\@supervisorBsDepartment{#1}}

\let\@supervisorCsLastname\@empty
\newcommand{\supervisorCsLastname}[1]{\def\@supervisorCsLastname{#1}}
\let\@supervisorCsFirstname\@empty
\newcommand{\supervisorCsFirstname}[1]{\def\@supervisorCsFirstname{#1}}
\let\@supervisorCsEmail\@empty
\newcommand{\supervisorCsEmail}[1]{\def\@supervisorCsEmail{#1}}
\let\@supervisorCsKTHID\@empty
\newcommand{\supervisorCsKTHID}[1]{\def\@supervisorCsKTHID{#1}}
\let\@supervisorCsOrganization\@empty
\newcommand{\supervisorCsOrganization}[1]{\def\@supervisorCsOrganization{#1}}
\let\@supervisorCsSchool\@empty
\newcommand{\supervisorCsSchool}[1]{\def\@supervisorCsSchool{#1}}
\let\@supervisorCsDepartment\@empty
\newcommand{\supervisorCsDepartment}[1]{\def\@supervisorCsDepartment{#1}}

\let\@supervisorDsLastname\@empty
\newcommand{\supervisorDsLastname}[1]{\def\@supervisorDsLastname{#1}}
\let\@supervisorDsFirstname\@empty
\newcommand{\supervisorDsFirstname}[1]{\def\@supervisorDsFirstname{#1}}
\let\@supervisorDsEmail\@empty
\newcommand{\supervisorDsEmail}[1]{\def\@supervisorDsEmail{#1}}
\let\@supervisorDsKTHID\@empty
\newcommand{\supervisorDsKTHID}[1]{\def\@supervisorDsKTHID{#1}}
\let\@supervisorDsOrganization\@empty
\newcommand{\supervisorDsOrganization}[1]{\def\@supervisorDsOrganization{#1}}
\let\@supervisorDsSchool\@empty
\newcommand{\supervisorDsSchool}[1]{\def\@supervisorDsSchool{#1}}
\let\@supervisorDsDepartment\@empty
\newcommand{\supervisorDsDepartment}[1]{\def\@supervisorDsDepartment{#1}}

\let\@supervisorEsLastname\@empty
\newcommand{\supervisorEsLastname}[1]{\def\@supervisorEsLastname{#1}}
\let\@supervisorEsFirstname\@empty
\newcommand{\supervisorEsFirstname}[1]{\def\@supervisorEsFirstname{#1}}
\let\@supervisorEsEmail\@empty
\newcommand{\supervisorEsEmail}[1]{\def\@supervisorEsEmail{#1}}
\let\@supervisorEsKTHID\@empty
\newcommand{\supervisorEsKTHID}[1]{\def\@supervisorEsKTHID{#1}}
\let\@supervisorEsOrganization\@empty
\newcommand{\supervisorEsOrganization}[1]{\def\@supervisorEsOrganization{#1}}
\let\@supervisorEsSchool\@empty
\newcommand{\supervisorEsSchool}[1]{\def\@supervisorEsSchool{#1}}
\let\@supervisorEsDepartment\@empty
\newcommand{\supervisorEsDepartment}[1]{\def\@supervisorEsDepartment{#1}}

\let\@examinersLastname\@empty
\newcommand{\examinersLastname}[1]{\def\@examinersLastname{#1}}
\let\@examinersFirstname\@empty
\newcommand{\examinersFirstname}[1]{\def\@examinersFirstname{#1}}
\let\@examinersEmail\@empty
\newcommand{\examinersEmail}[1]{\def\@examinersEmail{#1}}
\let\@examinersKTHID\@empty
\newcommand{\examinersKTHID}[1]{\def\@examinersKTHID{#1}}
\let\@examinersOrganization\@empty
\newcommand{\examinersOrganization}[1]{\def\@examinersOrganization{#1}}
\let\@examinersSchool\@empty
\newcommand{\examinersSchool}[1]{\def\@examinersSchool{#1}}
\let\@examinersDepartment\@empty
\newcommand{\examinersDepartment}[1]{\def\@examinersDepartment{#1}}

\let\@courseCycle\@empty
\newcommand{\courseCycle}[1]{\def\@courseCycle{#1}}
\let\@courseCode\@empty
\newcommand{\courseCode}[1]{\def\@courseCode{#1}}
\let\@courseCredits\@empty
\newcommand{\courseCredits}[1]{\def\@courseCredits{#1}}

\let\@degreeName\@empty
\newcommand{\degreeName}[1]{\def\@degreeName{#1}}
\let\@subjectArea\@empty
\newcommand{\subjectArea}[1]{\def\@subjectArea{#1}}

\let\@secondDegreeName\@empty
\newcommand{\secondDegreeName}[1]{\def\@secondDegreeName{#1}}
\let\@secondSubjectArea\@empty
\newcommand{\secondSubjectArea}[1]{\def\@secondSubjectArea{#1}}

% Data related to the oral presentation
\let\@presentationDateAndTimeISO\@empty
\newcommand{\presentationDateAndTimeISO}[1]{\def\@presentationDateAndTimeISO{#1}}
\let\@presentationLanguage\@empty
\newcommand{\presentationLanguage}[1]{\def\@presentationLanguage{#1}}
\let\@presentationRoom\@empty
\newcommand{\presentationRoom}[1]{\def\@presentationRoom{#1}}
\let\@presentationAddress\@empty
\newcommand{\presentationAddress}[1]{\def\@presentationAddress{#1}}
\let\@presentationCity\@empty
\newcommand{\presentationCity}[1]{\def\@presentationCity{#1}}

\let\@opponentsNames\@empty
\newcommand{\opponentsNames}[1]{\def\@opponentsNames{#1}}

% Data for DIVA National Subject Cateories fields
\let\@nationalsubjectcategories\@empty
\newcommand{\nationalsubjectcategories}[1]{\def\@nationalsubjectcategories{#1}}

% Data for UN's Sustainable Development Goals (SDGs)
\let\@SDGs\@empty
\newcommand{\SDGs}[1]{\def\@SDGs{#1}}

% Keywords
\let\@EnglishKeywords\@empty
\newcommand{\EnglishKeywords}[1]{\def\@EnglishKeywords{#1}}

\let\@SwedishKeywords\@empty
\newcommand{\SwedishKeywords}[1]{\def\@SwedishKeywords{#1}}

\ExplSyntaxOn
\cs_new_eq:NN \strcompare \str_if_eq:nnTF
\ExplSyntaxOff

%% Changed to use expl3 strcompare rather than xstring's IfEqCase, since the later is not expandable
%% The insight for this is from Enrico Gregorio - egreg's posing of 26 April 2016 at 
%% https://tex.stackexchange.com/questions/306484/how-do-i-perform-an-expandable-string-comparison

\newcommand{\InsertKeywords}[1]{
\strcompare{#1}{english}{\@EnglishKeywords}{}%
\strcompare{#1}{swedish}{\@SwedishKeywords}{}%
}




\let\@programcode\@empty
\newcommand{\programcode}[1]{%
\def\@programcode{#1}%
\edef\@prgmcode{\programmecodeToString{#1}}}

\let\@secondProgramcode\@empty
\newcommand{\secondProgramcode}[1]{%
\def\@secondProgramcode{#1}%
\edef\@secondprgmcode{\programmecodeToString{#1}}}

\let\@edprogram\@empty
\newcommand{\edprogram}[1]{%
\def\@edprogram{#1}}

\let\@secondedProgram\@empty
\newcommand{\secondedProgram}[1]{%
\def\@secondedProgram{#1}}

% to store the user's choice of copyright or copyleft
\let\@copyrightleft\@empty
% note that the command has to change the definition of \@copyrightleft globally, hence \gdef and not \def
\newcommand{\thesiscopyrightleft}[1]{\gdef\@copyrightleft{#1}}

% Get and store information about the series and the number within this series, i.e, TRITA numbers
%"Series": \{
%	"Title of series": "TRITA-ICT-EX",
%	"No. in series": "2019:00"
\let\@thesisSeries\@empty
\let\@thesisSeriesNumber\@empty
\newcommand{\trita}[2]{\def\@thesisSeries{#1}\def\@thesisSeriesNumber{#2}}

% for cover illustration
\let\@coverIllustration\@empty
\newcommand{\coverIllustration}[1]{\def\@coverIllustration{#1}}
\let\@coverIllustrationCredit\@empty
\newcommand{\coverIllustrationCredit}[1]{\def\@coverIllustrationCredit{#1}}

\makeatother

% Due to parsing problems with Overleaf's checking for balance of
% start and ends of beginning and ending use of "otherlanguage" and
% because babel and polyglossia & biblatex use different names for Norsk and German
% the following new commands have been added

\ifxeorlua
    \newcommand{\babelpolyLangStop}[1]{\end{#1}}
\else
    \newcommand{\babelpolyLangStop}[1]{\end{otherlanguage}}
\fi

\ifxeorlua
   \newcommand{\babelpolyLangStart}[1]{\begin{#1}}
\else
    \newcommand{\babelpolyLangStart}[1]{\begin{otherlanguage}{#1}}
\fi

\ExplSyntaxOn
% Note that this function returns the _string_ stored in the scontents buffer
\cs_new:Npn \qgetstored #1 #2
{
\__scontents_getfrom_seq:nn {#1} {#2}
}

\NewDocumentCommand{\replaceBS}{mm}
 {
  \tl_set_eq:NN #2 #1
  \tl_replace_all:NVn #2 \c_backslash_other_tl { ¢ }
 }
\tl_const:Nx \c_backslash_other_tl { \cs_to_str:N \\ }
\cs_generate_variant:Nn \tl_replace_all:Nnn { NV }

\ExplSyntaxOff

\makeatletter
\newcommand{\divainfo}[2]{
%%%%%%%%%%%%%%%%%%%%%%%%%%%%%%%%%%%%%%%%%%%%% for writing to a JSON file
\newwrite\jsonfile
\immediate\openout\jsonfile=fordiva.json
\immediate\write\jsonfile{\@charlb}
\immediate\write\jsonfile{"Author1": \@charlb \ifx\@authorsLastname\@empty%\relax
     \else
     "Last name": "\@authorsLastname",
     \fi
     \ifx\@authorsFirstname\@empty%\relax
     \else
     "First name": "\@authorsFirstname",
     \fi
     \ifx\@kthid\@empty%\relax
     \else
     "Local User Id": "\@kthid",
     \fi
     \ifx\@email\@empty%\relax
     \else
     "E-mail": "\@email",
     \fi
     \ifx\@authorsSchool\@empty%\relax
     \else
     "organisation": \@charlb"L1": "\@authorsSchool"
     \ifx\@authorsDepartment\@empty%\relax
     \else
     ,"L2": "\@authorsDepartment"
     \fi
     \@charrb
     \fi
\@charrb,}
\ifx\@secondAuthorsLastname\@empty\relax
\else
\immediate\write\jsonfile{
        "Author2": \@charlb \ifx\@secondAuthorsLastname\@empty%\relax
            \else
        "Last name": "\@secondAuthorsLastname",
        \fi
        \ifx\@secondAuthorsFirstname\@empty%\relax
            \else
        "First name": "\@secondAuthorsFirstname",
        \fi
        \ifx\@secondkthid\@empty%\relax
            \else
        "Local User Id": "\@secondkthid",
        \fi
        \ifx\@secondemail\@empty%\relax
            \else
        "E-mail": "\@secondemail",
        \fi
        \ifx\@secondAuthorsSchool\@empty%\relax
        \else
        "organisation": \@charlb"L1": "\@secondAuthorsSchool"
        \ifx\@secondAuthorsDepartment\@empty%\relax
        \else
        ,"L2": "\@secondAuthorsDepartment"
        \fi
        \@charrb
		\fi
\@charrb,}
\fi % end of conditional for secondAuthorsLastname
\immediate\write\jsonfile{\ifx\@courseCycle\@empty%\relax
\else
"Cycle": "\@courseCycle",
\fi
\ifx\@courseCode\@empty%\relax
\else
"Course code": "\@courseCode",
\fi
\ifx\@courseCredits\@emptyv
\else
"Credits": "\@courseCredits",
\fi
}
\immediate\write\jsonfile{"Degree1": \@charlb"Educational program": "\@prgmcode"% 
       \ifx\@programcode\@empty%\relax
       \else
       ,"programcode": "\@programcode"
       \fi
        \ifx\@degreeName\@empty%\relax
       \else
       ,"Degree": "\@degreeName"
       \fi
        \ifx\@subjectArea\@empty%\relax
       \else
       ,"subjectArea": "\@subjectArea"
       \fi
       \@charrb,
       }
       \ifx\@secondProgramcode\@empty%\relax           %% If there is a second program code then add this element
       \else
        \immediate\write\jsonfile{
        "Degree2": \@charlb"Educational program": "\@secondprgmcode" %
        \ifx\@secondProgramcode\@empty%\relax
         \else
         ,"programcode": "\@secondProgramcode"
         \fi
         \ifx\@secondDegreeName\@empty%\relax
        \else
        ,"Degree": "\@secondDegreeName"
        \fi
            \ifx\@secondSubjectArea\@empty%\relax
        \else
         ,"SubjectArea": "\@secondSubjectArea"
         \fi
         \@charrb,\\
       }
     \fi
\immediate\write\jsonfile{"Title": \@charlb
       "Main title": "\@title",
       \ifx\@subtitle\@empty%\relax
       \else
       "Subtitle": "\@subtitle",
       \fi
       \ifinswedish
       "Language": "swe"
       \else
       "Language": "eng"
       \fi
       \@charrb,
    "Alternative title": \@charlb
    "Main title": "\@alttitle",
    \ifx\@altsubtitle\@empty%\relax
    \else
    "Subtitle": "\@altsubtitle",
    \fi
    \ifinswedish
    "Language": "eng"
    \else
    "Language": "swe"
    \fi
    \@charrb,
}
\ifx\@titleInPlainText\@empty%\relax
\else
\immediate\write\jsonfile{"Title in plain text": \@charlb
       "Main title": "\@titleInPlainText",
       \ifx\@subtitleInPlainText\@empty%\relax
       \else
       "Subtitle": "\@subtitleInPlainText",
       \fi
       \ifinswedish
       "Language": "swe"
       \else
       "Language": "eng"
       \fi
       \@charrb,
    "Alternative title in plain text": \@charlb
    "Main title": "\@alttitleInPlainText",
    \ifx\@altsubtitleInPlainText\@empty%\relax
    \else
    "Subtitle": "\@altsubtitleInPlainText",
    \fi
    \ifinswedish
    "Language": "eng"
    \else
    "Language": "swe"
    \fi
    \@charrb,
}
\fi
\ifx\@supervisorAsLastname\@emptyv
        \else
\immediate\write\jsonfile{"Supervisor1": \@charlb "Last name": "\@supervisorAsLastname",
    \ifx\@supervisorAsFirstname\@emptyv
    \else
    "First name": "\@supervisorAsFirstname",
    \fi
    \ifx\@supervisorAsKTHID\@empty%\relax
    \else
    "Local User Id": "\@supervisorAsKTHID",
    \fi
    \ifx\@supervisorAsEmail\@empty%\relax
    \else
    "E-mail": "\@supervisorAsEmail",
    \fi
    \ifx\@supervisorAsOrganization\@empty
    \ifx\@supervisorAsSchool\@empty%\relax
    \else
    "organisation": \@charlb"L1": "\@supervisorAsSchool"
    \ifx\@supervisorAsDepartment\@empty%\relax
    \else
    ,"L2": "\@supervisorAsDepartment"
    \fi
    \@charrb
    \fi
    \else
    "Other organisation": "\@supervisorAsOrganization"
    \fi
    \@charrb,
}
\fi
\ifx\@supervisorBsLastname\@empty%\relax
\else
\immediate\write\jsonfile{"Supervisor2": \@charlb "Last name": "\@supervisorBsLastname",
    \ifx\@supervisorBsFirstname\@empty%\relax
    \else
    "First name": "\@supervisorBsFirstname",
    \fi
    \ifx\@supervisorBsKTHID\@empty%\relax
    \else
    "Local User Id": "\@supervisorBsKTHID",
    \fi
    \ifx\@supervisorBsEmail\@empty%\relax
    \else
    "E-mail": "\@supervisorBsEmail",
    \fi
    \ifx\@supervisorBsOrganization\@empty
    \ifx\@supervisorBsSchool\@empty%\relax
    \else
    "organisation": \@charlb"L1": "\@supervisorBsSchool"
    \ifx\@supervisorBsDepartment\@empty%\relax
    \else
    ,"L2": "\@supervisorBsDepartment"
    \fi
    \@charrb
    \fi
    \else
    "Other organisation": "\@supervisorBsOrganization"
    \fi
    \@charrb,
}
\fi
\ifx\@supervisorCsLastname\@empty%\relax
\else
\immediate\write\jsonfile{"Supervisor3": \@charlb "Last name": "\@supervisorCsLastname",
    \ifx\@supervisorCsFirstname\@empty%\relax
    \else
    "First name": "\@supervisorCsFirstname",
    \fi
    \ifx\@supervisorCsKTHID\@empty%\relax
    \else
    "Local User Id": "\@supervisorCsKTHID",
    \fi
    \ifx\@supervisorCsEmail\@empty%\relax
    \else
    "E-mail": "\@supervisorCsEmail",
    \fi
    \ifx\@supervisorCsOrganization\@empty
    \ifx\@supervisorCsSchool\@empty%\relax
    \else
    "organisation": \@charlb"L1": "\@supervisorCsSchool"
    \ifx\@supervisorCsDepartment\@empty%\relax
    \else
    ,"L2": "\@supervisorCsDepartment"
    \fi
    \@charrb
    \fi
    \else
    "Other organisation": "\@supervisorCsOrganization"
    \fi
    \@charrb,
}
\fi
\ifx\@supervisorDsLastname\@empty%\relax
\else
\immediate\write\jsonfile{"Supervisor4": \@charlb "Last name": "\@supervisorDsLastname",
    \ifx\@supervisorDsFirstname\@empty%\relax
    \else
    "First name": "\@supervisorDsFirstname",
    \fi
    \ifx\@supervisorDsKTHID\@empty%\relax
    \else
    "Local User Id": "\@supervisorDsKTHID",
    \fi
    \ifx\@supervisorDsEmail\@empty%\relax
    \else
    "E-mail": "\@supervisorDsEmail",
    \fi
    \ifx\@supervisorDsOrganization\@empty
    \ifx\@supervisorDsSchool\@empty%\relax
    \else
    "organisation": \@charlb"L1": "\@supervisorDsSchool"
    \ifx\@supervisorDsDepartment\@empty%\relax
    \else
    ,"L2": "\@supervisorDsDepartment"
    \fi
    \@charrb
    \fi
    \else
    "Other organisation": "\@supervisorDsOrganization"
    \fi
    \@charrb,
}
\fi
\ifx\@supervisorEsLastname\@empty%\relax
\else
\immediate\write\jsonfile{"Supervisor5": \@charlb "Last name": "\@supervisorEsLastname",
    \ifx\@supervisorEsFirstname\@empty%\relax
    \else
    "First name": "\@supervisorEsFirstname",
    \fi
    \ifx\@supervisorEsKTHID\@empty%\relax
    \else
    "Local User Id": "\@supervisorEsKTHID",
    \fi
    \ifx\@supervisorEsEmail\@empty%\relax
    \else
    "E-mail": "\@supervisorEsEmail",
    \fi
    \ifx\@supervisorEsOrganization\@empty
    \ifx\@supervisorEsSchool\@empty%\relax
    \else
    "organisation": \@charlb"L1": "\@supervisorEsSchool"
    \ifx\@supervisorEsDepartment\@empty%\relax
    \else
    ,"L2": "\@supervisorEsDepartment"
    \fi
    \@charrb
    \fi
    \else
    "Other organisation": "\@supervisorEsOrganization"
    \fi
    \@charrb,
}
\fi
\ifx\@examinersLastname\@empty%\relax
\else
\immediate\write\jsonfile{"Examiner1": \@charlb "Last name": "\@examinersLastname",
    \ifx\@examinersFirstname\@empty%\relax
    \else
    "First name": "\@examinersFirstname",
    \fi
    \ifx\@examinersKTHID\@empty%\relax
    \else
    "Local User Id": "\@examinersKTHID",
    \fi
    \ifx\@examinersEmail\@empty%\relax
    \else
    "E-mail": "\@examinersEmail",
    \fi
    \ifx\@examinersOrganization\@empty
    \ifx\@examinersSchool\@empty%\relax
    \else
    "organisation": \@charlb"L1": "\@examinersSchool"
    \ifx\@examinersDepartment\@empty%\relax
    \else
    ,"L2": "\@examinersDepartment"
    \fi
    \@charrb
    \fi
    \else
    "Other organisation": "\@examinersOrganization"\@charrb
    \fi
    \@charrb,
}
\fi
\immediate\write\jsonfile{\ifx\@hostcompany\@empty%\relax
\else
"Cooperation": \@charlb "Partner_name": "\@hostcompany"\@charrb,
\fi
}
\ifx\@nationalsubjectcategories\@empty%\relax
\else
\immediate\write\jsonfile{"National Subject Categories": "\@nationalsubjectcategories",
}
\fi
\ifx\@SDGs\@empty%\relax
\else
\immediate\write\jsonfile{"SDGs": "\@SDGs",
}
\fi
% \ifx\@hostorganization\@empty\relax\else Host organization: \@hostorganization\\\fi
% "\pageref{pg:lastPageofPreface},\pageref{pg:lastPageofMainmatter}" \@charrb,
\immediate\write\jsonfile{"Other information": \@charlb"Year": "\the\year", "Number of pages": 
    "\getpagerefnumber{pg:lastPageofPreface},\getpagerefnumber{pg:lastPageofMainmatter}" \@charrb,
}
\ifx\@copyrightleft\@empty
\immediate\write\jsonfile{"Copyrightleft": "None",
}
\else
\immediate\write\jsonfile{"Copyrightleft": "\@copyrightleft",
}
\fi
\immediate\write\jsonfile{"Series": \@charlb
    \ifx\@thesisSeries\@empty\relax,
        \else
        	"Title of series": "\@thesisSeries"
        	\ifx\@thesisSeriesNumber\@empty\relax
            \else
            , "No. in series": "\@thesisSeriesNumber"
             \fi
    \fi
    \@charrb,
}
\ifx\@opponentsNamea\@empty\relax
\else
\immediate\write\jsonfile{"Opponents": \@charlb "Name": "\@opponentsNames"\@charrb,
}
\fi
\ifx\@presentationDateAndTimeISO\@empty%\relax,
\else
\immediate\write\jsonfile{"Presentation": \@charlb "Date": "\@presentationDateAndTimeISO"
        \ifx\@presentationLanguage\@empty%\relax,
        \else
        ,"Language": "\@presentationLanguage"
        \fi
        \ifx\@presentationRoom\@empty%
        \else
        ,"Room": "\@presentationRoom"
        \fi
        \ifx\@presentationAddress\@empty%\relax,
        \else
        ,"Address": "\@presentationAddress"
        \fi
        \ifx\@presentationCity\@empty%\relax,
        \else
        ,"City": "\@presentationCity"
        \fi
        \@charrb,
}
\fi
\immediate\write\jsonfile{"Number of lang instances": "\countsc{lang}",}
% pdflatex cannot handle the abstracts or the keywords in the code below,
% it generates a "! TeX capacity exceeded, sorry [input stack size=5000]." 
% error message.
\ifxeorlua
\immediate\write\jsonfile{"abstracts": \@charlb}
 \foreach \i in {1,...,\countsc{lang}} {
 %\getstored[\i]{lang}
  \newcommand{\templangs}{\qgetstored{\i}{lang}}
  \tokenize{\lang}{\templangs}
  \immediate\write\jsonfile{"\lang": €€€€}
  \immediate\write\jsonfile{"\qgetstored{\i}{abstracts}"}
  \immediate\write\jsonfile{€€€€,}
}
\immediate\write\jsonfile{\@charrb,} % end the abstracts dict

\immediate\write\jsonfile{"keywords": \@charlb}
\foreach \i in {1,...,\countsc{lang}} {
    \newcommand{\templangs}{\qgetstored{\i}{lang}}
    \tokenize{\lang}{\templangs}
    \newcommand{\fee}{\qgetstored{\i}{keywords}}
    % Because \qgetstored{\i}{keywords} returned a string,
    % we need to convert this string into tokens so it can be evaluated
    % we do this using the function \tokenisze from the xstring package
    \tokenize{\fumble}{\fee}
    \immediate\write\jsonfile{"\lang": €€€€}
    %\immediate\write\jsonfile{"\fee"}  %% This would write out the contents of the stored keywords
    \immediate\write\jsonfile{"\fumble"} %% This expands the macros in the keywords and write the results
    \immediate\write\jsonfile{€€€€,}
}
    \immediate\write\jsonfile{\@charrb} % end the keywords dict
\fi % end of the ifxeorlua

    \immediate\write\jsonfile{\@charrb} % end the JSON dict

    \closeout\jsonfile
    
     %%%%%% Now make the For DiVA page
    \thispagestyle{empty}
    \sffamily\tiny
    \begin{flushleft}
    \{\\
        "Author1": \{ \ifx\@authorsLastname\@empty\relax
     \else
     "Last name": "\@authorsLastname",\\
     \fi
     \ifx\@authorsFirstname\@empty\relax
     \else
     "First name": "\@authorsFirstname",\\
     \fi
     \ifx\@kthid\@empty\relax
     \else
     "Local User Id": "\@kthid",\\
     \fi
     \ifx\@email\@empty\relax
     \else
     "E-mail": "\@email",\\
     \fi
     \ifx\@authorsSchool\@empty\relax
     \else
     "organisation": \{"L1": "\@authorsSchool",\\
     \ifx\@authorsDepartment\@empty\relax
     \else
     ,"L2": "\@authorsDepartment"
     \fi
     \}\\
     \fi
    \},\\
    \ifx\@secondAuthorsLastname\@empty\relax
    \else
        "Author2": \{ \ifx\@secondAuthorsLastname\@empty\relax
            \else
        "Last name": "\@secondAuthorsLastname",\\
        \fi
        \ifx\@secondAuthorsFirstname\@empty\relax
            \else
        "First name": "\@secondAuthorsFirstname",\\
        \fi
        \ifx\@secondkthid\@empty\relax
            \else
        "Local User Id": "\@secondkthid",\\
        \fi
        \ifx\@secondemail\@empty\relax
            \else
        "E-mail": "\@secondemail",\\
        \fi
        \ifx\@secondAuthorsSchool\@empty\relax
        \else
        "organisation": \{"L1": "\@secondAuthorsSchool",\\
        \ifx\@secondAuthorsDepartment\@empty\relax
        \else
        "L2": "\@secondAuthorsDepartment"
        \fi
        \}\\
		\fi
    \},\\
    \fi
    \ifx\@courseCycle\@empty\relax
    \else
    "Cycle": "\@courseCycle",\\
    \fi
    \ifx\@courseCode\@empty\relax
    \else
    "Course code": "\@courseCode",\\
    \fi
    \ifx\@courseCredits\@empty\relax
    \else
    "Credits": "\@courseCredits",\\
     \fi

       "Degree1": \{"Educational program": "\@prgmcode"\\ % 
       \ifx\@programcode\@empty\relax
       \else
       ,"programcode": "\@programcode"\\
       \fi

        \ifx\@degreeName\@empty\relax
       \else
       ,"Degree": "\@degreeName"\\
       \fi
        \ifx\@subjectArea\@empty\relax
       \else
       ,"subjectArea": "\@subjectArea"\\
       \fi
       \},\\
       \ifx\@secondProgramcode\@empty\relax            %% If there is a second program code then add this element
       \else
        "Degree2": \{"Educational program": "\@secondprgmcode"\\
        \ifx\@secondProgramcode\@empty\relax
         \else
         ,"programcode": "\@secondProgramcode"\\
         \fi
         \ifx\@secondDegreeName\@empty\relax
        \else
        ,"Degree": "\@secondDegreeName"\\
        \fi
            \ifx\@secondSubjectArea\@empty\relax
        \else
         ,"SubjectArea": "\@secondSubjectArea"\\
         \fi
         \},\\
       \fi
       "Title": \{\\
       "Main title": "\@title",\\
       \ifx\@subtitle\@empty\relax
       \else
       "Subtitle": "\@subtitle",\\
       \fi
       \ifinswedish
       "Language": "swe"
       \else
       "Language": "eng"
       \fi
       \},\\
    "Alternative title": \{\\
    "Main title": "\@alttitle",\\
    \ifx\@altsubtitle\@empty\relax
    \else
    "Subtitle": "\@altsubtitle",\\
    \fi
    \ifinswedish
    "Language": "eng"\\
    \else
    "Language": "swe"\\
    \fi
    \},\\
     \ifx\@supervisorAsLastname\@empty\relax
        \else
    "Supervisor1": \{ "Last name": "\@supervisorAsLastname",\\
    \ifx\@supervisorAsFirstname\@empty\relax
    \else
    "First name": "\@supervisorAsFirstname",\\
    \fi
    \ifx\@supervisorAsKTHID\@empty\relax
    \else
    "Local User Id": "\@supervisorAsKTHID",\\
    \fi
    \ifx\@supervisorAsEmail\@empty\relax
    \else
    "E-mail": "\@supervisorAsEmail",\\
    \fi
    \ifx\@supervisorAsOrganization\@empty
    \ifx\@supervisorAsSchool\@empty\relax
    \else
    "organisation": \{"L1": "\@supervisorAsSchool",\\
    \ifx\@supervisorAsDepartment\@empty\relax
    \else
    "L2": "\@supervisorAsDepartment"
    \fi
    \}\\
    \fi
    \else
    "Other organisation": "\@supervisorAsOrganization"\\
    \fi
    \},\\
    \fi
    \ifx\@supervisorBsLastname\@empty\relax
    \else
    "Supervisor2": \{ "Last name": "\@supervisorBsLastname",\\
    \ifx\@supervisorBsFirstname\@empty\relax
    \else
    "First name": "\@supervisorBsFirstname",\\
    \fi
    \ifx\@supervisorBsKTHID\@empty\relax
    \else
    "Local User Id": "\@supervisorBsKTHID",\\
    \fi
    \ifx\@supervisorBsEmail\@empty\relax
    \else
    "E-mail": "\@supervisorBsEmail",\\
    \fi
    \ifx\@supervisorBsOrganization\@empty
    \ifx\@supervisorBsSchool\@empty\relax
    \else
    "organisation": \{"L1": "\@supervisorBsSchool",\\
    \ifx\@supervisorBsDepartment\@empty\relax
    \else
    "L2": "\@supervisorBsDepartment"
    \fi
    \}\\
    \fi
    \else
    "Other organisation": "\@supervisorBsOrganization"\\
    \fi
    \},\\
    \fi
    \ifx\@supervisorCsLastname\@empty\relax
    \else
    "Supervisor3": \{ "Last name": "\@supervisorCsLastname",\\
    \ifx\@supervisorCsFirstname\@empty\relax
    \else
    "First name": "\@supervisorCsFirstname",\\
    \fi
    \ifx\@supervisorCsKTHID\@empty\relax
    \else
    "Local User Id": "\@supervisorCsKTHID",\\
    \fi
    \ifx\@supervisorCsEmail\@empty\relax
    \else
    "E-mail": "\@supervisorCsEmail",\\
    \fi
    \ifx\@supervisorCsOrganization\@empty
    \ifx\@supervisorCsSchool\@empty\relax
    \else
    "organisation": \{"L1": "\@supervisorCsSchool",\\
    \ifx\@supervisorCsDepartment\@empty\relax
    \else
    "L2": "\@supervisorCsDepartment"
    \fi
    \}\\
    \fi
    \else
    "Other organisation": "\@supervisorCsOrganization"\\
    \fi
    \},\\
    \fi
    \ifx\@supervisorDsLastname\@empty\relax
    \else
    "Supervisor4": \{ "Last name": "\@supervisorDsLastname",\\
    \ifx\@supervisorDsFirstname\@empty\relax
    \else
    "First name": "\@supervisorDsFirstname",\\
    \fi
    \ifx\@supervisorDsKTHID\@empty\relax
    \else
    "Local User Id": "\@supervisorDsKTHID",\\
    \fi
    \ifx\@supervisorDsEmail\@empty\relax
    \else
    "E-mail": "\@supervisorDsEmail",\\
    \fi
    \ifx\@supervisorDsOrganization\@empty
    \ifx\@supervisorDsSchool\@empty\relax
    \else
    "organisation": \{"L1": "\@supervisorDsSchool",\\
    \ifx\@supervisorDsDepartment\@empty\relax
    \else
    "L2": "\@supervisorDsDepartment"
    \fi
    \}\\
    \fi
    \else
    "Other organisation": "\@supervisorDsOrganization"\\
    \fi
    \},\\
    \fi
    \ifx\@supervisorEsLastname\@empty\relax
    \else
    "Supervisor5": \{ "Last name": "\@supervisorEsLastname",\\
    \ifx\@supervisorEsFirstname\@empty\relax
    \else
    "First name": "\@supervisorEsFirstname",\\
    \fi
    \ifx\@supervisorEsKTHID\@empty\relax
    \else
    "Local User Id": "\@supervisorEsKTHID",\\
    \fi
    \ifx\@supervisorEsEmail\@empty\relax
    \else
    "E-mail": "\@supervisorEsEmail",\\
    \fi
    \ifx\@supervisorEsOrganization\@empty
    \ifx\@supervisorEsSchool\@empty\relax
    \else
    "organisation": \{"L1": "\@supervisorEsSchool",\\
    \ifx\@supervisorEsDepartment\@empty\relax
    \else
    "L2": "\@supervisorEsDepartment"
    \fi
    \}\\
    \fi
    \else
    "Other organisation": "\@supervisorEsOrganization"\\
    \fi
    \},\\
    \fi
    \ifx\@examinersLastname\@empty\relax
        \else
    "Examiner1": \{ "Last name": "\@examinersLastname",\\
    \ifx\@examinersFirstname\@empty\relax
    \else
    "First name": "\@examinersFirstname",\\
    \fi
    \ifx\@examinersKTHID\@empty\relax
    \else
    "Local User Id": "\@examinersKTHID",\\
    \fi
    \ifx\@examinersEmail\@empty\relax
    \else
    "E-mail": "\@examinersEmail",\\
    \fi
    \ifx\@examinersOrganization\@empty
    \ifx\@examinersSchool\@empty\relax
    \else
    "organisation": \{"L1": "\@examinersSchool",\\
    \ifx\@examinersDepartment\@empty\relax
    \else
    "L2": "\@examinersDepartment"
    \fi
    \}\\
    \fi
    \else
    "Other organisation": "\@examinersOrganization"\}\\
    \fi
    \},\\
    \fi
    \ifx\@hostcompany\@empty\relax
    \else
    "Cooperation": \{ "Partner\_name": "\@hostcompany"\},\\
    \fi
    \ifx\@nationalsubjectcategories\@empty\relax
    \else
    \textquotedbl\relax National Subject Categories\textquotedbl\relax: \textquotedbl\relax\@nationalsubjectcategories\textquotedbl\relax,\\
    \fi
    \ifx\@SDGs\@empty%\relax
    \else
    \textquotedbl\relax SDGs\textquotedbl\relax:
    \textquotedbl\relax\@SDGs\textquotedbl\relax,\\
    \fi
    % \ifx\@hostorganization\@empty\relax\else Host organization: \@hostorganization\\\fi
    "Other information": \{"Year": "\the\year", "Number of pages": \char`\"\pageref{#1},\pageref{#2}\char`\"
    \},\\
    \ifx\@copyrightleft\@empty
    "Copyrightleft": "None",\\
    \else
    "Copyrightleft": "\@copyrightleft",\\
    \fi
    "Series": \{
    \ifx\@thesisSeries\@empty\relax
        \else
        	"Title of series": "\@thesisSeries"
        	\ifx\@thesisSeriesNumber\@empty\relax
            \else
            , "No. in series": "\@thesisSeriesNumber"
             \fi
    \fi
    \},\\
        \ifx\@opponentsNamea\@empty\relax
        \else
    "Opponents": \{ "Name": "\@opponentsNames"\},\\
       \fi
        \ifx\@presentationDateAndTimeISO\@empty\relax
        \else
       "Presentation": \{ "Date": "\@presentationDateAndTimeISO"\par
        \ifx\@presentationLanguage\@empty\relax
        \else
        ,"Language":"\@presentationLanguage"\par
        \fi
        \ifx\@presentationRoom\@empty\relax
        \else
        ,"Room": "\@presentationRoom"\par
        \fi
        \ifx\@presentationAddress\@empty\relax
        \else
        ,"Address": "\@presentationAddress"\par
        \fi
        \ifx\@presentationCity\@empty\relax
        \else
        ,"City": "\@presentationCity"
        \fi
        \},\par
    \fi
    "Number of lang instances": \textquotedbl\relax\countsc{lang}\textquotedbl\relax,\\
    \foreach \i in {1,...,\countsc{lang}} {
      "Abstract[\getstored[\i]{lang}]": €€€€\\
      \ifxeorlua
       % if using XeLaTeX or LuaLaTeX get the unprocessed saved abstracts (i.e., in latex form)
      %\typestoredx{\i}{abstracts}
      \expandafter\typestored\expandafter[\the\numexpr\i\relax]{abstracts}
      \else
      % if using pdflatex get the processed saved abstracts - otherwise it will have problems with utf-8 characters
      \getstored[\i]{abstracts}
      \fi
    }
    \foreach \i in {1,...,\countsc{lang}} {
      €€€€,\\
      "Keywords[\getstored[\i]{lang}]": €€€€\\
      \getstored[\i]{keywords}
      €€€€,\\
    }
   \}

    % Attach the fordiva.json file
    \IfFileExists{fordiva.json}{
    \attachfile[description={fordiva.json}]{fordiva.json}
    }{}
   \end{flushleft}
\clearpage


 
    %% experiment with dc:description
\newcommand{\myabstract}{abstract: unknown}
 \foreach \i in {1,...,\countsc{lang}} {
  \newcommand{\templangs}{\qgetstored{\i}{lang}}
  \tokenize{\lang}{\templangs}
  \typeout{templangs}
    \typeout{\templangs}
   \typeout{raw qget}
   \typeout{\qgetstored{1}{abstracts}}
   \StrSubstitute{q}{q}{\qgetstored{\i}{abstracts}}[\@myabstract]
    \typeout{myabstract}
    \typeout{\@myabstract}
    \replaceBS{\@myabstract}{\@myabstract}

    \typeout{myabstract after substitution}
    \typeout{\myabstract}
    %\StrSubstitute{\@myabstract}{¢item}{*char'}[\@myabstract]
    
    \replaceBS{\@myabstract}{\@myabstract}
    \typeout{myabstract after cleanup}
    \typeout{\@myabstract}
   %\StrLeft{\@myabstract}{100}[\@myabstract]
    \renewcommand{\myabstract}{\@myabstract}
    %% Convert from three-letter language code to short language code
    \IfEqCase{\templangs}{
    {eng}{\XMPLangAlt{en-US}{pdfsubject={\myabstract}}}%
    {swe}{\XMPLangAlt{sv}{pdfsubject={\myabstract}}}%
    {fre}{\XMPLangAlt{fr-FR}{pdfsubject={\myabstract}}}%
    {spa}{\XMPLangAlt{es-ES}{pdfsubject={\myabstract}}}%
    {ita}{\XMPLangAlt{it}{pdfsubject={\myabstract}}}%
    {nno}{\XMPLangAlt{nn}{pdfsubject={\myabstract}}}%
    {nob}{\XMPLangAlt{nb}{pdfsubject={\myabstract}}}%
    {nor}{\XMPLangAlt{no}{pdfsubject={\myabstract}}}%
    {ger}{\XMPLangAlt{de-DE}{pdfsubject={\myabstract}}}%
    {dan}{\XMPLangAlt{da}{pdfsubject={\myabstract}}}%
    {dut}{\XMPLangAlt{nl}{pdfsubject={\myabstract}}}%
    {est}{\XMPLangAlt{et}{pdfsubject={\myabstract}}}%
    {ukr}{\XMPLangAlt{uk}{pdfsubject={\myabstract}}}%
    {lat}{\XMPLangAlt{la}{pdfsubject={\myabstract}}}%
    {fin}{\XMPLangAlt{fi}{pdfsubject={\myabstract}}}%
    {ice}{\XMPLangAlt{is}{pdfsubject={\myabstract}}}%
    {yid}{\XMPLangAlt{yi}{pdfsubject={\myabstract}}}%
    {lit}{\XMPLangAlt{lt}{pdfsubject={\myabstract}}}%
    {pol}{\XMPLangAlt{pl}{pdfsubject={\myabstract}}}%
    {por}{\XMPLangAlt{pt}{pdfsubject={\myabstract}}}%
    {per}{\XMPLangAlt{fa}{pdfsubject={\myabstract}}}%
    {rum}{\XMPLangAlt{ro}{pdfsubject={\myabstract}}}%
    {hin}{\XMPLangAlt{hi}{pdfsubject={\myabstract}}}%
    {hun}{\XMPLangAlt{hu}{pdfsubject={\myabstract}}}%
    {bul}{\XMPLangAlt{bg}{pdfsubject={\myabstract}}}%
    {alb}{\XMPLangAlt{sq}{pdfsubject={\myabstract}}}%
    {cat}{\XMPLangAlt{ca}{pdfsubject={\myabstract}}}%
    {chi}{\XMPLangAlt{zh}{pdfsubject={\myabstract}}}%
    {ara}{\XMPLangAlt{ar}{pdfsubject={\myabstract}}}%
    {bel}{\XMPLangAlt{be}{pdfsubject={\myabstract}}}%
    {rus}{\XMPLangAlt{ru}{pdfsubject={\myabstract}}}%
    {vie}{\XMPLangAlt{vi}{pdfsubject={\myabstract}}}%
    {scc}{\XMPLangAlt{sr}{pdfsubject={\myabstract}}}% deprecated B code
    {cze}{\XMPLangAlt{cs}{pdfsubject={\myabstract}}}%
    [\typeout{unknown three-letter language code}]
    }

}

}

\makeatother

% to define a command\B to bold font entries in a table
% based on https://tex.stackexchange.com/questions/469559/bold-entries-in-table-with-s-column-type
\usepackage{etoolbox}
\renewcommand{\bfseries}{\fontseries{b}\selectfont}
\robustify\bfseries
\newrobustcmd{\B}{\bfseries}
